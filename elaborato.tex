%%%%%%%%%%%%%%%%%%%%%%%%%%%%%%%%%%%%%%%%%%%%%%%%%%%%%%%%%%%%%%%%%%%%%%%%%%%
%                                                                         %
%			TEMPLATE LATEX PER TESI                                       %
%			______________                                                %
%                                                                         %
%           Ultima revisione: 13 aprile 2023                              %
%           Revisori: G.Presti; L.A.Ludovico; F. Avanzini; M. Tiraboschi, %
%                     Marco Aceti                                         %
%                                                                         %
%%%%%%%%%%%%%%%%%%%%%%%%%%%%%%%%%%%%%%%%%%%%%%%%%%%%%%%%%%%%%%%%%%%%%%%%%%%

\documentclass[12pt,italian]{report}
\usepackage{template}


% CORSO DI LAUREA:
\def\myCDL{Corso di Laurea in Informatica}

% TITOLO TESI:
\def\myTitle{Trasporto ferroviario e Open Data}

% AUTORE:
\def\myName{\textbf{Marco Aceti}}
\def\myMat{Matr.\ 963032}

% RELATORE E CORRELATORE:
\def\myRefereeA{Prof.\ Andrea Trentini}
%\def\myRefereeB{Prof. Federico Avanzini}
%\def\myRefereeC{Prof. Giorgio Presti} % commentare in caso di un solo correlatore

% ANNO ACCADEMICO
\def\myYY{2022-2023}

% Il seguente comando introduce un elenco delle figure dopo l'indice (facoltativo)
%\figurespagetrue

% Il seguente comando introduce un elenco delle tabelle dopo l'indice (facoltativo)
%\tablespagetrue

%
%			PREAMBOLO
%			Inserire qui eventuali package da includere o definizioni di comandi personalizzati
%

% Package di formato
\usepackage[a4paper]{geometry}		% Formato del foglio
\usepackage[italian]{babel}			% Supporto per l'italiano
\usepackage[utf8]{inputenc}			% Supporto per UTF-8
%\usepackage[a-1b]{pdfx}			% File conforme allo standard PDF-A (obbligatorio per la consegna)

% Package per la grafica
\usepackage{graphicx}				% Funzioni avanzate per le immagini
\usepackage{hologo}					% Bibtex logo with \hologo{BibTeX}

% Package tipografici
\usepackage{amssymb,amsmath,amsthm} % Simboli matematici
\usepackage{listings}				% Scrittura di codice

% Package ipertesto
\usepackage{url}					% Visualizza e rendere interattii gli URL
\usepackage{hyperref}				% Rende interattivi i collegamenti interni


\begin{document}
	
	% Creazione automatica del frontespizio
	\frontespizio
	\beforepreface
	
	% DEDICA
	\hfill
	\begin{minipage}{15cm}
		\hfill
		\begin{minipage}[t]{11cm}
			\raggedleft \large
			{
				\sl
				
				Nei confronti di Trenord \\
				sono state usate espressioni violente come \\
				``disastro, incubo, degrado, punizione''. \\
				\bigskip
				Abbiamo sempre rispettato le opinioni di tutti,\\
				ma dinanzi a tanta accanita disinformazione crediamo doveroso rivolgerci direttamente a voi, clienti e viaggiatori, per darvi i dati reali di quanto accaduto ed offrirvi le corrette spiegazioni.
				
				\bigskip
			}
		\end{minipage} \\
			\raggedleft \large
			%\textbf{-- Paolo Garavaglia} \\
			%\textrm{Direttore Comunicazioni e Relazioni Esterne, Trenord} \\
			
			\textbf{Lettera ai clienti Trenord} \\
			Milano, 2 febbraio 2022
	\end{minipage}
	
	\prefacesection{Ringraziamenti}
	Questa sezione, facoltativa, contiene i ringraziamenti.
	
	\afterpreface
	
	\chapter{Introduzione}
	\label{cap:introduzione}
	
	Introduzione
	
	\bibliographystyle{unsrt}
	\bibliography{bibliografia}
	\addcontentsline{toc}{chapter}{Bibliografia}

\end{document}
