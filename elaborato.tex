%%%%%%%%%%%%%%%%%%%%%%%%%%%%%%%%%%%%%%%%%%%%%%%%%%%%%%%%%%%%%%%%%%%%%%%%%%%
%                                                                         %
%			TEMPLATE LATEX PER TESI                                       %
%			______________                                                %
%                                                                         %
%           Ultima revisione: 13 aprile 2023                              %
%           Revisori: G.Presti; L.A.Ludovico; F. Avanzini; M. Tiraboschi, %
%                     Marco Aceti                                         %
%                                                                         %
%%%%%%%%%%%%%%%%%%%%%%%%%%%%%%%%%%%%%%%%%%%%%%%%%%%%%%%%%%%%%%%%%%%%%%%%%%%

\documentclass[12pt,italian]{report}
\usepackage{template}


% CORSO DI LAUREA:
\def\myCDL{Corso di Laurea in Informatica}

% TITOLO TESI:
\def\myTitle{Trasporto ferroviario e Open Data}

% AUTORE:
\def\myName{\textbf{Marco Aceti}}
\def\myMat{Matr.\ 963032}

% RELATORE E CORRELATORE:
\def\myRefereeA{Prof.\ Andrea Trentini}

% ANNO ACCADEMICO
\def\myYY{2022-2023}

% Il seguente comando introduce un elenco delle figure dopo l'indice (facoltativo)
%\figurespagetrue

% Il seguente comando introduce un elenco delle tabelle dopo l'indice (facoltativo)
%\tablespagetrue

%
%			PREAMBOLO
%

% Package di formato
\usepackage[a4paper]{geometry}		% Formato del foglio
\usepackage[italian]{babel}			% Supporto per l'italiano
\usepackage[utf8]{inputenc}			% Supporto per UTF-8
%\usepackage[a-1b]{pdfx}			% File conforme allo standard PDF-A (obbligatorio per la consegna)

% Package per la grafica
\usepackage{graphicx}				% Funzioni avanzate per le immagini
\usepackage{hologo}					% Bibtex logo with \hologo{BibTeX}

% Citazioni e bibliografia
\usepackage[style=italian]{csquotes}
\usepackage[backend=biber,style=numeric,sorting=none]{biblatex}
\addbibresource{bibliografia.bib}

% Package tipografici
\usepackage{amssymb,amsmath,amsthm} % Simboli matematici
\usepackage{listings}				% Scrittura di codice

% Package ipertesto
\usepackage{url}					% Visualizza e rendere interattii gli URL
\usepackage{hyperref}				% Rende interattivi i collegamenti interni


\begin{document}
	
	% Creazione automatica del frontespizio
	\frontespizio
	\beforepreface
	
	% DEDICA
	\hfill
	\begin{minipage}{15cm}
		\hfill
		\begin{minipage}[t]{11cm}
			\raggedleft \large
			{
				\sl
				
				Nei confronti di Trenord \\
				sono state usate espressioni violente come \\
				``disastro, incubo, degrado, punizione''. \\
				\bigskip
				Abbiamo sempre rispettato le opinioni di tutti,\\
				ma dinanzi a tanta accanita disinformazione crediamo doveroso rivolgerci direttamente a voi, clienti e viaggiatori, per darvi i dati reali di quanto accaduto ed offrirvi le corrette spiegazioni.
				
				\bigskip
			}
		\end{minipage} \\
			\raggedleft \large
			%\textbf{-- Paolo Garavaglia} \\
			%\textrm{Direttore Comunicazioni e Relazioni Esterne, Trenord} \\
			
			\textbf{Trenord, Lettera ai clienti} \\
			Milano, 2 febbraio 2022
	\end{minipage}
	
	% \prefacesection{Ringraziamenti}
	% Questa sezione, facoltativa, contiene i ringraziamenti.
	
	\afterpreface
	
	\chapter{Introduzione}
	Ad oggi in Italia non esistono Open Data generalizzati sulla \textit{reale} performance del \textbf{trasporto pubblico ferroviario}; l'accesso a tali metriche è ostacolato da \textbf{barriere tecnologiche e legali}, spesso ingiustificate. 
	Le poche e frammentate statistiche pubblicamente disponibili su eventi come \textit{ritardi} o \textit{cancellazioni} sono spesso arbitrariamente aggregate, impedendo l'uso di strumenti di analisi avanzati. \\
	
	\textit{Come può un Cittadino, quindi, valutare l'operato delle imprese ferroviarie e degli enti committenti senza l'accesso ai \textbf{dati reali} sul servizio?} \\
	
	Il lavoro di tesi si articola sull’idea di \textbf{preservare i dati istantanei} della circolazione ferroviaria dalla piattaforma ViaggiaTreno per produrre Open Data storici, \textit{machine-readable} e di qualità.
	Inoltre, verranno proposte a fini dimostrativi alcune analisi dei dati raccolti.
	
	\section{Il trasporto ferroviario in Italia}
	
	Sul territorio nazionale sono previsti due soggetti distinti che operano sulla ferrovia:
	\begin{itemize}
		\item le \textbf{imprese ferroviarie}, ovvero
		\textcquote[art.\ 2, comma 1, lettera a)]{Dlgs112}{qualsiasi impresa pubblica o privata
			titolare di una licenza, la cui attività principale consiste nella
			prestazione di servizi per il trasporto sia di merci sia di persone
			per ferrovia e che garantisce obbligatoriamente la trazione};
		\item i \textbf{gestori dell'infrastruttura}, ovvero
		\textcquote[art.\ 2, comma 1, lettera b)]{Dlgs112}{qualsiasi organismo o impresa responsabili dell'esercizio, della manutenzione e del rinnovo dell'infrastruttura ferroviaria di una rete nonché della partecipazione al suo sviluppo come stabilito dallo Stato nell'ambito della sua politica generale sullo sviluppo e sul finanziamento dell'infrastruttura}.
	\end{itemize}

	\subsection{Le Società sul territorio}
	
	Attualmente sono certificate 18 \textbf{imprese ferroviarie} per il servizio passeggeri \cite{ElencoIF}, tra cui Trenitalia S.p.A., Trenord S.r.l.\ e Trenitalia Tper S.C.a.R.L.
	Il principale \textbf{gestore dell'infrastruttura} è Rete Ferroviaria Italiana S.p.A.\ (RFI, 16829 km \cite{RfiKm}), seguito da Ferrovie Emilia Romagna (343 km \cite{FerKm}) e Ferrovienord (330 km \cite{FerNordKm}).
	Il gruppo \textbf{Ferrovie dello Stato Italiane S.p.A.} è una partecipata del Ministero dell'Economia e delle Finanze \cite{MefGruppoFS} e controlla totalmente alcune delle Società già citate, come Trenitalia S.p.A.\ e Rete Ferroviaria Italiana S.p.A.\ \cite{ControllateFS}. 
	
	\paragraph{Lombardia}
	
	Trenord S.r.l.\ è una \textit{joint venture}\footnote{Contratto di accordo tra due o più imprese al fine di raggiungere un obiettivo comune} di Trenitalia (al 50\%) e il gruppo FNM S.p.A.\ (FerrovieNord Milano, al 50\%) \cite{TrenordChiSiamo}. 
	Quest'ultimo è posseduto da Regione Lombardia (al 57,57\%) e da Trenitalia stessa (al 14,74\%); il restante 27,69\% è quotato in Borsa \cite{BorsaItalianaFNM}.
	La bizzarra composizione societaria cela una totale partecipazione di Trenitalia S.p.A.\ del $\sim 57\%$, di Regione Lombardia del $\sim 28$\% e del restante $\sim 15\%$ da parte di azionisti privati.
	
	L'infrastruttura è gestita sia da RFI\ (1740 km \cite{RfiKm}) che da Ferrovienord (controllata da FNM, 330 km \cite{FerNordKm}).
	
	\paragraph{Emilia-Romagna}
	
	La Trenitalia Tper S.C.a.R.L.\ è una società consortile partecipata da Trenitalia S.p.A.\ (al 70\%) e TPER S.p.a.\ (Trasporto Passeggeri Emilia-Romagna, al 30\%) \cite{NascitaTper}; quest'ultima è a sua volta ripartita tra diversi soggetti pubblici dell'Emilia Romagna \cite{SociTper}.
	
	Similmente alla Lombardia, l'infrastruttura è gestita sia da RFI (1319 km \cite{RfiKm}) che da FER S.r.l.\ (controllata dalla Regione \cite{FerChiSiamo}, 343 km \cite{FerKm}).
	
	\subsection{Il servizio pubblico e il mercato}
	
	La fornitura del servizio di trasporto ferroviario passeggeri è costituita da \textbf{offerte a mercato} (treni ad alta velocità) e \textbf{contratti di servizio pubblico}, stipulati dalle imprese ferroviarie con lo Stato (per le tratte a media e lunga percorrenza) o con le Regioni (per le connessioni regionali e interregionali) \cite[vedi][paragrafo \textit{``Gli obblighi di servizio pubblico e i contratti di servizio''}]{CameraTrasportoFerroviario}.
	
	Il servizio pubblico può essere oggetto di \textbf{compensazioni economiche} da parte dell'Ente, ma queste devono essere commisurate in base a dei \textbf{parametri ben definiti} \cite[art.\ 4 comma 1]{Reg1370}.
	Fino al 25 dicembre 2023 permane la possibilità delle autorità di affidare direttamente i contratti di servizio pubblico, senza gara \cite[art.\ 8, comma 2, lettera iii)]{Reg1370}.
	
	Il quadro normativo evidenzia la \textbf{funzione pubblica} del servizio di trasporto ferroviario regionale e interregionale, esente dalle normali logiche di libero mercato ma \textcquote{CameraTrasportoFerroviario}{\textit{funzionale ad assicurare \textbf{il diritto costituzionale alla mobilità}}}.
	La forte responsabilità politica degli enti committenti nel definire concessioni, contratti di servizio e compensazioni ha un grande impatto sulla qualità del trasporto ferroviario, sia in termini di programmazione che di \textit{performance}.

	\printbibliography
	\addcontentsline{toc}{chapter}{Bibliografia}

\end{document}
